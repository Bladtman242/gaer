\documentclass[a4paper, titlepage]{report}
\usepackage[utf8]{inputenc}
\usepackage{courier} % Required for the courier font
\usepackage[bookmarks]{hyperref}
\usepackage{amsfonts}

%redefine percentage sign to be a little smaller
\let\oldpct\%
\renewcommand{\%}{\scalebox{.9}{\oldpct}}

\begin{document}

\title{Artificial Life Individual Assignment}
\author{
	Sigurt Bladt Dinesen
	\\\texttt{sidi@itu.dk}
}

\maketitle

\tableofcontents
\chapter{Evolutionary Computation}
This chapter details my solution to \texttt{Option 2}, using a genetic algorithm
to solve an instance of the Traveling Salesman Problem (TSP).

The source code is provided \href{https://github.com/Bladtman242/GA-TSP}{here}
\footnote{https://github.com/Bladtman242/GA-TSP }. The code is writen as a
literate haskell program, and can as such be read instead of this part of the
report.

There are four main points to define a genetic algorithm. Here are the
definitions I used to this lab:
\begin{description}
\item[Genotype and phenotype]
The genotype is just an ordered list of vertexes (city ids).
The interpretation of the genotype (the phenotype) is a path through the graph
representing our instance of TSP, progressing through the vertices in the
genome, returning the first vertex in the end.

\item[Mutation]
The step that takes one generation of genomes, and creates a
new generation of genomes. In this report, mutation is done by sexual
reproduction between two genomes, by random crossover, producing only one child.

I experimented with random mutation as well, but did
not achieve any better results.

\item[Fitness]
Determines the \textit{evolutionary} value of a genome, for use in
\texttt{selection}. The obvious choice for a (fitness) function here is the
travel-cost of the phenotype. Neither mutation nor the definition of the genome
ensures the \textit{validity} of a genome in this solution, i.e. a genome could
look like $\left[1,1,1,1,1,1,1,1,1,1\right]$, which is not just a \textit{bad}
solution, but not a solution at all. So the fitness function must steer the
selection away from such genomes. To achieve this, another term is included in
the fitness function: An exponential function (for smoothness) of the Jaccard
distance between the genome, and the list of all vertexes
($\left[1,2,3,4,5,6,7,8,9,10\right]$). 
The final fitness function of a genome $g$ is defined as

	$$2^{J(g) \cdot T} + pathlength(g)$$

Where $J$ is one minus the Jaccard distance between $g$ and the set of all
vertices, and $T$ is a large constant term to ensure that non-zero Jaccard
distances are weighed heavier than any path length.

\item[Selection]
The step that takes one generation of genomes, and decides who (or whose children)
gets to be part of the next generation of genomes.
In this project, selection is done by taking the best half of the population, and
letting it, and their children, be the new population.
This does not preserve diversity very well, but seems to produce good results
nonetheless.
\end{description}

This genetic algorithm converges on a solution after ca. 15 generations. After ca. 30, the worst and best solutions in the genome remain the same, and are of similar fitness.


\end{document}
